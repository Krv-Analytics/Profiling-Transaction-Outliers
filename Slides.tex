%!TEX TS-program = lualatex
%!TEX encoding   = UTF-8 Unicode

\documentclass[aspectratio=169]{beamer}

\usepackage{tikz}
\usepackage{svg}
\usepackage{subfiles}
\usepackage{hyperref}
\usepackage{hyperxmp}
\usepackage{qrcode}
\usepackage{figures}
\usepackage[export]{adjustbox}


\hypersetup{
  pdfcopyright = {Copyright (C) 2023 by Krv Analytics.
    All rights reserved. Do not distribute.},
  pdflicenseurl = {http://latex-community.org/license/}}
  
\tikzstyle{box} = [rectangle, rounded corners, minimum width=1.5cm, minimum height=1cm,text centered, draw=black, fill=white]
\tikzstyle{box2} = [rectangle, rounded corners, minimum width=0.5cm, minimum height=0.5cm,text centered, draw=black, fill=white]
\tikzstyle{arrow} = [thick,->,>=stealth]

\usetheme{purus}

%%%%%%%%%%%%%%%%%%%%%%%%%%%%%%%%%%%%%%%%%%%%%%%%%%%%%%%%%%%%%%%%%%%%%%%%
% Title
%%%%%%%%%%%%%%%%%%%%%%%%%%%%%%%%%%%%%%%%%%%%%%%%%%%%%%%%%%%%%%%%%%%%%%%%

% 
\title[Transaction Anomaly Detection]{Identification and Classification of Transaction Anomalies}
\institute{Applying isolation outlier detection and topological profiling to GL Transaction data.}

%%%%%%%%%%%%%%%%%%%%%%%%%%%%%%%%%%%%%%%%%%%%%%%%%%%%%%%%%%%%%%%%%%%%%%%%
% Slides
%%%%%%%%%%%%%%%%%%%%%%%%%%%%%%%%%%%%%%%%%%%%%%%%%%%%%%%%%%%%%%%%%%%%%%%%

\begin{document}
  \begin{frame}[noframenumbering, plain]
  \setlength {\marginparwidth }{2cm}
    \titlepage
  \end{frame}

\begin{frame}{Some Terminology}
  \begin{itemize}
    \item \textit{Outlier}: an extrememal transaction (ie differeing extensively from a reference set of transactions) 
    \item \textit{Anomaly}: an unexpected pattern of transactions
    \item \textit{Suspect}: a transaction that has been flagged as potentially erroneous or malicious
    \item \textit{Guilty Transaction}: a confirmed erroneous or malicious transaction (and intuitively there also exist innocent suspects).
  \end{itemize}
\end{frame}

\begin{frame}{The Goal}
We split the obvious end of identifying all guilty transactions into two simpler subgoals: 
\begin{enumerate}
  \item Identifing a set of suspects 
  \item Profiling Suspects 
\end{enumerate}
Performing these two tasks will generate a case against suspects. Like a competent prosecutor, only convincing cases should 
lead to an accusal. 
\end{frame}
  \begin{frame}{Building a Good Case}
    Sticking with our metaphor, the classic "motive, means, opportunity" evaluation
    will translate to "difference and danger" in our data-driven perspective. 
    \begin{itemize}
      \item \textit{Difference}: also to be thought of as distance, refers to the necessary 
      separation between guilty and innocent transactions.  
      \item \textit{Danger}: contexualized properties that identify erratic, peculiar, 
    unexplained or risky behavior    
    \end{itemize}
    
  \end{frame}

  \begin{frame}{Establishing Difference}
    As a start, we construct a suspect list comprised of elements 
  \end{frame}

\end{document}
